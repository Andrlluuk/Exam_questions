
\documentclass[10pt]{amsart}
%\usepackage[cp1251]{inputenc}
\usepackage[english,russian]{babel}
\usepackage{amsmath}
\usepackage{amssymb}
\usepackage{amsfonts}
\usepackage{chngcntr,color}
%\usepackage[hyper]{amsbib}
\usepackage{ulem}
\usepackage[hidelinks]{hyperref}
\usepackage[mathscr]{euscript}
\usepackage{enumitem}
\usepackage{cite}
\usepackage{bbm} % We need this for pretty Indicator \mathbbm{1}.
\usepackage[left=25mm, top=20mm, right=10mm, bottom=10mm, nohead, nofoot]{geometry}
    \usepackage{graphicx} % для картинок


\renewcommand\theequation{\arabic{section}.\arabic{equation}}
\newcommand{\ra}{\rightarrow}
\newcommand{\p}[1]{{\mathbf{P}} \left( \, #1 \, \right) }
\newcommand{\px}[1]{{\mathbf{P}_x}\!\left( \, #1 \, \right) }
\newcommand{\e}{{\mathbf{E}} }
\newcommand{\lk}{«}
\newcommand{\pk}{»}
\newcommand{\lm}{(\lambda,\mu)}
\newcommand{\sk}[1]{\left[ #1 \right]}
\newcommand{\skk}[1]{\left\{ #1 \right\}}
\newcommand{\lef}{\left(}
\newcommand{\rig}{\right)}
\newcommand{\skl}[1]{\left\langle #1 \right\rangle}

\renewcommand{\le}{\leqslant}
\renewcommand{\ge}{\geqslant}

 \def\lms{\mathop{\overline{\lim}}\limits}

\def\lmi{\mathop{\underline{\lim}}\limits}

\def\v{\varepsilon}

\def\d{\delta}
\def\a{\alpha}
\def\b{\beta}
\def\g{\gamma}
\def\l{\lambda}
\def\m{\mu}
\def\({\left(}
\def\){\right)}

\def\({\left(}
\def\){\right)}

\def\[{\left[}
\def\]{\right]}


\def\|{\left|}
\def\|{\right|}



\def\SL{\left\{}
\def\SP{\right\}}

\def\dd{{\boldsymbol{\delta}}}
\def\aa{{\boldsymbol{\alpha}}}
\def\bb{{\boldsymbol{\beta}}}
\def\gg{{\boldsymbol{\gamma}}}
\def\ll{{\boldsymbol{lambda}}}
\def\mm{{\boldsymbol{\mu}}}
\def\0{{\boldsymbol{0}}}


\def\n{\nu}
\def\t{\tau}
\def\x{\xi}
\def\z{\zeta}
\def\zz{{\boldsymbol{\zeta}}}
\def\e{\eta}
\def\h{\theta}
\def\ll{{\boldsymbol{\lambda}}}
\def\mm{{\boldsymbol{\mu}}}
\def\xx{{\boldsymbol{\xi}}}
\def\ppsi{{\boldsymbol{\psi}}}
\def\LL{{\boldsymbol{\Lambda}}}



\def\DD{Д\ о\ к\ а\ з\ а\ т\ е\ л\ ь\ с\ т\ в\ о}
\def\D{\mathbb D}
\def\C{\mathbb C}
\def\R{\mathbb R}
\def\Z{\mathbb Z}
\def\V{\mathbb V}


\def\r{\rho}
\def\x{\xi}


%    \raisebox{-1pt}[0pt][0pt]{\includegraphics[width=0.02\linewidth]{3.png}}
\begin{document}


\thispagestyle{empty}

\begin{center}
    {\Large Программа по теории вероятностей 2020}
\end{center}


\begin{enumerate}
\item[Глава 1.] Случайные события и их вероятности \\
\begin{enumerate}
\item[\S\, 1.1.] Элементарная теория вероятностей \\
3.5. Опр. (пространства элементарных исходов), примеры пространств элементарных исходов и событий. \\
3. Опр. (операций над событиями, несовместных событий). \\
\item[\S\, 1.2.] Модель дискретной вероятности \\
3. Аксиомы классической вероятностной модели. Примеры случайных экспериментов, которые одновременно могут быть описаны классической и неклассической вероятностной схемой. \\
3. Аксиомы дискретной вероятности. Примеры. \\
\item[\S\, 1.3.] Геометрические вероятности. \\
3.  Аксиомы геометрической вероятностной модели. Примеры. \\
4. Пример (парадокс Бертрана). \\ 

\item[\S\, 1.4.] Аксиоматическое построение теории вероятностей \\ 
3.  Опр.( сигма – алгебры). Примеры сигма-алгебр. \\
 \raisebox{-1pt}[0pt][0pt]{\includegraphics[width=0.02\linewidth]{4.png}} Свойства сигма-алгебр с доказательствами (пустое мно-во, конечное объединение, счетное пересечение). Формула двойственности. \\
3. Опр. (борелевской сигма-алгебра). \\
3. Опр. (вероятностной меры). \\
4.  Свойства вероятностной меры с доказательствами (вероятность пустого мн-ва, дизъюнктного объединения, дополнения, объединения двух мн-в, монотонность). \\
5. Свойства  вероятностной меры с доказательством (вероятность объединения счетного набора, непрерывность вер. меры, формула включения/исключения). \\
3. Опр. (вероятностного пространства). Примеры. \\

\item[\S\, 1.5.] Условная вероятность, независимость \\
3. Опр. (условной вероятности).  Свойство (о перемножении вероятностей). \\ 
3. Опр. (двух независимых событий). \\
4.5. Свойства независимых событий с доказательством (несовместность, условная вероятность, \\ теоретико-множественные операции). \\
3. Опр. (событий, независимых в совокупности). \\
4. Пример (Бернштейна). \\


    \item[\S\, 1.6.] Схема Бернулли \\
3. Опр. (Схемы Бернулли). Примеры экспериментов со схемой Бернулли. \\
3. Теорема (формула Бернулли). (без док-ва). \\
4.  Теорема (формула Бернулли). Доказательство. \\
4. Теорема Пуассона для схемы Бернулли. Доказательство. \\
4. Теорема (номер первого успешного испытания в схеме Бернулли). Доказательство. \\
5. Теорема (полиномиальная схема). Доказательство. \\
Задача. К.1.3
Задача. К.1.5
Задача. К.1.31
Задача. К.1.34
Задача. К.1.37
Задача. К.1.42

    
    
\item[\S\, 1.7.] Формул полной вероятности \\
3. Опр. (полной группы событий). \\
4. Теорема (формула полной вероятности). Доказательство. \\
4.5. Пример (задача о разорении для двух игроков при помощи ФПВ). \\ 
4. Теорема (формула Байеса). Доказательство. \\
3. Примеры случайных экспериментов, описываемых с помощью ФПВ и формулы Байеса. \\
\end{enumerate}
\item[Глава 2.]  Случайные величины и их распределения \\
    
\begin{enumerate}
\item[\S\, 2.1.]  Случайные величины \\
3. Опр. (случайной величины). \\ 
4. Примеры вероятностных пространств и функций, которые являются или не являются случайными величинами с доказательством. \\

    
\item[\S\, 2.2.] Распределения случайных величин \\
3. Опр. (распределения случайной величины). Примеры распределений, как вероятностных мер. \\
3. Опр. (дискретного распределения). Примеры. \\
3. Опр. (абсолютно непрерывного распределения). Примеры. \\
4. Теорема (о плотности). Доказательство. \\
3. Опр. (сингулярного распределения). Примеры. \\
3. Опр. (смешанного распределения). Примеры. \\


\item[\S\, 2.3.] Функция распределения \\
    
3.5. Опр. (функции распределения). Примеры. \\
3. Свойства функций распределения (без док-ва). \\
4. Свойства функций распределения с доказательствами. \\
3. Теорема (о классе функций распределения) (без док-ва)\\


\item[\S\, 2.4.] Примеры распределений \\
    
3. Опр. (вырожденного распределения). Пример случайных экспериментов и случайной величины с этим распределением. \\
3.2. Опр. (распределения Бернулли).  Пример случайных экспериментов и случайной величины с этим распределением. \\
3. Опр. (биномиального распределения). Доказательство, что это действительно распределение. Пример случайных экспериментов и случайной величины с этим распределением. \\     
3. Опр. (геометрического распределения). Доказательство, что это действительно распределение. Свойство нестарения. Пример случайных экспериментов и случайной величины с этим распределением. \\      
3. Опр. (распределения Пуассона). Доказательство, что это действительно распределение. \\
3. Опр. (гипергеометрического распределения). Доказательство, что это действительно распределение. Пример случайных экспериментов и случайной величины с этим распределением. \\      
3. Опр. (равномерного распределения). Доказательство, что это действительно распределение. Вычисление функции распределения. Пример случайных экспериментов и случайной величины с этим распределением. \\
3. Опр. (показательного распределения). Доказательство, что это действительно распределение. Свойство нестарения. Вычисление функции распределения. \\  
3.2. Опр. (нормального (гауссовского) распределения). Свойство линейных преобразований с доказательством. \\
Вопрос на 5. Вычисление интеграла от плотности нормального распределения.  Свойства нормального распределения с доказательством (лин. преобр., равенства для $\Phi_{0,1},$ правило трех сигм). \\
3. Опр. (гамма распределения). Доказательство, что это действительно распределение. \\
3. Опр. (распределения Коши). Доказательство, что это действительно распределение. Вычисление функции распределения. \\
3. Опр. (распределения Парето). Доказательство, что это действительно распределение. Вычисление функции распределения. \\
3. Опр. (логнормального распределения). Вычисление плотности. \\
4. Пример сингулярного распределения (лестница Кантора). \\
3. Опр. (смеси распределений). Пример задания смеси двойной рандомизацией. \\

\item[\S\, 2.5.] Преобразования случайных величин \\
3. Замечание (об измеримости преобразования случайной величины). \\
4. Теорема (о плотности и линейном преобразование случайных величин). Доказательство. \\
3. Опр. (квантили для непрерывной функции распределения). \\
3. Опр. (квантили в общей случае). \\
3. Опр. (медианы).   Опр. (моды). \\
5. Теорема (о квантилях и линейном преобазование случайных величин, обобщенная обратная функция). Доказательство. \\
5. Теорема (о квантильном преобразование). Доказательство. \\

    
\item[\S\, 2.6.] Многомерные распределения \\
3. Опр. (случайного вектора). \\
3. Опр. (совместного распределения и  совместной функции распределения). \\
4. Свойства совместной функции распределения. Доказательство. \\
3.5. Опр. (дискретного многомерного распределения). Свойства. Примеры. \\
3. Опр. (многомерного абсолютно непрерывного распределения). \\
4. Нахождение маргинальных плотностей по многомерной плотности. \\
3. Опр. (многомерного равномерного распределения). \\
3. Опр. (многомерного нормального распределения). Вид плотности для многомерного стандартного нормального вектора. \\ 
3. Опр. (многомерного сингулярного распределения). Пример. \\
3. Опр. (распределения Дирихле). \\
3. Опр. (независимых случайных величин). \\
3. Теорема (об эквивалентных определениях независимости) (без док-ва). \\
5. Теорема (об эквивалентных определениях независимости). Доказательство. \\
4.Теорема (о сохранение независимости при преобразованиях). Доказательство. \\ 
4. Теорема (свертка для дискретных). Доказательство. \\
4. Теорема (свертка для дискретных).  Доказательство. \\
Задача. К.2.13
Задача. К.2.33
Задача. К.2.36
Задача. К.2.38
Задача. К.2.41
Задача. К.2.43
\end{enumerate}
    
    
\item[Глава 3.] Числовые характеристики распределений \\
    
\begin{enumerate}
\item[\S\, 3.1.] Интеграл по вероятностной мере. Математическое ожидание. \\
3.1. Опр. (простой случайной величины). \\
3. Опр. (математического ожидания для простой случайной величины). \\
4. Свойства математического ожидания для  простых случайных величин с доказательством. \\
3. Опр. (математического ожидания для простой случайной величины по событию). \\
4. Лемма (о приближение случайной величины простыми). Доказательство. \\
4. Лемма (о единственности предела для математического ожидания от простых). Доказательство. \\ 
3. Опр. (математического ожидания). \\
4. Основные свойства математического ожидания с доказательством (Однородность, монотонность, нер-во треугольника, аддитивность). \\
5.  Свойство счетной аддитивности математического ожидания. Доказательство. \\
3. Свойство  математического ожидания для независимых случайных величин (без док-ва). \\
4. Свойство  математического ожидания для независимых случайных величин. Доказательство. \\
3. Замечание (о вычисление  математического ожидания для дискретных, для а.н.р). \\
3. Замечание (о вычисление  математического ожидания для преобразований случайных величин (одномерных и многомерных преобразований)). \\
5. Теорема (о свертке для произвольных распределений). Доказательство. Следствие об а.н.р. суммы. \\ 
4. Примеры вычисления математического ожидания (Бернулли, биномиальное (двумя способами,нормальное). \\


\item[\S\, 3.2.] Моменты высшего порядка \\
3. Опр. ($k-$ого момента, $k$-ого центрального момента). Формулы для вычисления у дискретного и а.н.р. \\
4. Теорема (о существование математического ожидания меньших порядков). Доказательство. \\


\item[\S\, 3.3.] Моментные Неравенства \\
3.3. Теорема (неравенство Маркова). (без док-ва) \\
4. Теорема (неравенство Маркова). Доказательство. \\
4. Следствие из неравенства Маркова о распределение неотрицательной с.в. с нулевым МО. Доказательство. \\
3. Теорема (обобщенное неравенство Чебышёва) (без док-ва). \\
4. Теорема (обобщенное неравенство Чебышёва). Доказательство. \\
4. Теорема (неравенство Коши-Буняковского). Доказательство. \\
3. Теорема (неравенство Йенсена) (без док-ва). \\
4. Теорема (неравенство Йенсена). Доказательство. \\

\item[\S\, 3.4.] Дисперсия \\
3. Опр. (дисперсии, стандартного отклонения). \\ 
4. Свойства дисперсии с доказательством (альтернативный способ вычисления, критерий вырожденности, линейные преобр. одной случайной величины ). \\ 
4. Свойства дисперсии с доказательством (дисперсия суммы независимых с.в., оптимизационная задача). \\ 
4. Классическое неравенство Чебышёва. Доказательство. \\
4. Примеры вычисления дисперсии (Бернулли, биномиального и нормального). \\
 
        
        
\item[\S\, 3.5.] Коэффициент корреляции \\
3. Опр. (ковариации двух случайных величин). \\
3. Свойства ковариации (без док-ва). \\
4. Свойства ковариации с доказательством. \\
3. Опр. (коэффициента корреляции). \\
3. Свойства коэффициента корреляции (без док-ва). \\
5. Свойства коэффициента корреляции с доказательством. \\



       
        

\item[\S\, 3.6.] Матрица ковариации \\
3. Опр. (математического ожидания для случайного вектора и случайной матрицы). \\
4. Свойства многомерного математического ожидания (линейность, произведение независимых матриц). Доказательство. \\ 
3. Опр. (матрицы ковариации случайного вектора). \\
4. Свойства матрицы ковариации (при линейном преобразование, для суммы независимых случайных векторов). Доказательство. \\


\item[\S\, 3.7.] Многомерное нормальное распределение \\

3. Теорема (о линейном преобразование для нормального вектора) (без док-ва). \\
5. Теорема (о линейном преобразование для нормального вектора). Доказательство. \\
3. Следствие (о независимости и корреляции для нормального вектора) (без док-ва). \\ 
4.4. Следствие (о независимости и корреляции для нормального вектора). Доказательство. \\ 
3. Следствие (о независимости и  ортогональном преобразование нормального вектора) (без док-ва). \\ 
4. Следствие (о независимости и  ортогональном преобразование нормального вектора). Доказательство. \\ 
4. Контрпример не нормального вектора с нормальными одномерными компонентами. Доказательство. \\ 
Задача. К.3.5
Задача. К.3.9
Задача. К.3.14
Задача. К.3.21
Задача. К.3.29
Задача. К.3.35
Задача. К.3.41
Задача. К.3.47

\item[\S\, 3.8.] Копулы \\
        
3. Опр. (копулы). \\
4. Теорема (Шкляра). Доказательство в непрерывном случае. \\
4. Примеры базовых копул. \\
4. Теорема (неравенства Frechet-Hoeffding). Доказательство. \\
3. Опр (носителя случайной величины). \\
3. Опр. (неубывающего множества). \\
5. Теорема (о правой границе неравенства Frechet-Hoeffding).  Доказательство. \\
4. Следствие (об идеальной зависимости). Доказательство. \\
5. Теорема (об инвариантности копулы при строго возрастающем преобразовании). Доказательство. \\
3. Опр. (коэффициента корреляции Спирмена). \\
3. Опр. (коэффициента корреляции Кендалла). \\
5. Свойства коэффициентов корреляции Спирмена и Кендалла. Доказательство. \\ 
3. Опр. (гауссовской копулы). \\
3. Опр. (коэффициентов экстремальной зависимости). \\
4. Лемма (о коэффициентах экстремальной зависимости в непрерывном случае). Доказательство. \\
 
 
 
 
\item[\S\, 3.9.] Условное математическое ожидание \\
3. Опр. (условного математического ожидания). \\
3. Теорема о существование УМО (без док-ва). \\
4. Свойства УМО с доказательством (УМО константы, УМО от измеримой с.в., монотонность,  линейность, неравенство треугольника,  аналог формулы полной вероятности). \\
5. Свойства УМО с доказательством (УМО по более бедной сигма алгебре, вынос измеримой с.в. ). \\
5. Теорема об ортогональной проекции. Доказательство. \\
3. Лемма (вычисление УМО для а.н.р.)  (без док-ва). \\
4. Лемма (вычисление УМО для а.н.р.). Доказательство. \\
3. Лемма (вычисление УМО для дискретных). (без док-ва). \\
4. Лемма (вычисление УМО для дискретных). Доказательство. \\
5. Теорема (УМО для гауссовских векторов). Доказательство. \\

\end{enumerate}
    
    
\item[Глава 4.] Сходимость случайных величин и распределений. Предельные теоремы \\
\begin{enumerate}
\item[\S\, 4.1.] Сходимость последовательностей случайных величин \\
5. Теорема (Бореля-Кантелли). Доказательство. \\
3. Опр. (сходимости почти наверное). \\
3. Опр. (сходимости по вероятности). \\
3. Опр. (слабой сходимости). \\
4. Замечание (о равномерной сходимости, если ф.р. непрерывна). Доказательство. \\
3. Опр. (сходимости в среднеквадратическом). \\
4. Лемма (критерий сходимости п.н.). Доказательство. \\
3. Теорема (п.н. vs по вероятности)  (без док-ва). \\
5. Теорема (п.н. vs по вероятности). Доказательство. \\
3. Теорема (по вероятности vs слабая) (без док-ва). \\  
5. Теорема (по вероятности vs слабая). Доказательство. \\  
3. Теорема (с.к.с. vs p vs п.н.) (без док-ва). \\ 
4. Теорема (с.к.с. vs p vs п.н.). Доказательство. \\ 

        
        
\item[\S\, 4.2.] Свойства сходимостей \\
        
3. Теорема (критерий сходимости по распределению) (без док-ва). \\
4. Лемма  (сходимость при непрерывных преобразованиях). Доказательство. \\
4.4. Лемма (сходимость и арифметические операции). Доказательство. \\
3. Теорема Слуцкого (без док-ва). \\
5. Теорема Слуцкого. Доказательство. \\
3. Опр. (равномерной интегрируемости). \\
3. Теорема (критерий сходимости математических ожиданий) (без док-ва). \\
4. Теорема Лебега. Доказательство. \\

         
    
\item[\S\, 4.3.] Характеристические функции \\
3. Опр. (характеристической функции). \\
3. Замечание (о вычисление и существование х.ф.). \\
3.4. Свойства характеристических функций (значение в нуле, линейное преобразование, сумма независимых, гладкость в нуле) (без док-ва). \\
5. Свойства характеристических функций с доказательством (значение в нуле, линейное преобразование, сумма независимых, гладкость в нуле). \\
4. Примеры вычисления характеристических функций (вырожденное, Пуассона, нормальное). \\
5. Теорема (формула обращения). Доказательство. \\
4. Замечание (почему так важна формула обращения для характеристических функций?). \\
3. Следствие об устойчивости по суммированию (без док-ва). \\
4. Следствие об устойчивости по суммированию. Доказательство. \\
3. Теорема о непрерывном соответствие (без док-ва). \\
3.3. Теорема (закон больших чисел Хинчина) (без док-ва). \\
4. Теорема (закон больших чисел Хинчина).  Доказательство. \\
3. Теорема (закон больших чисел Колмогорова) (без док-ва). \\
5. Теорема (закон больших чисел Колмогорова). Доказательство достаточности при 4-ом моменте. \\

 

\item[\S\, 4.4.] Центральная предельная теорема \\
3.3. Теорема (центральная предельная теорема) (без док-ва). \\
4. Теорема (центральная предельная теорема). Доказательство. \\        
4. Следствие (из ЦПТ). \\
3.4. Теорема (неравенство Берри-Эссеена) (без док-ва). \\
4. Замечание (о неулучшаемости неравенства Берри-Эссеена). \\
5. Теорема (оценка точности в теореме Пуассона). Доказательство. \\
    









    \end{enumerate}
    

\end{enumerate}

\end{document}