

\documentclass[10pt]{article}

%\usepackage[cp1251]{inputenc}

\usepackage[english,russian]{babel}

\usepackage{amsmath}

\usepackage{amssymb}

\usepackage{amsfonts}

\usepackage{chngcntr,color}

%\usepackage[hyper]{amsbib}

\usepackage{ulem}

\usepackage[hidelinks]{hyperref}

\usepackage[mathscr]{euscript}

\usepackage{enumitem}

\usepackage{cite}

\usepackage{bbm} % We need this for pretty Indicator \mathbbm{1}.

\usepackage[left=25mm, top=20mm, right=10mm, bottom=10mm, nohead]{geometry}

    \usepackage{graphicx} % для картинок





\renewcommand\theequation{\arabic{section}.\arabic{equation}}

\newcommand{\ra}{\rightarrow}

\newcommand{\p}[1]{{\mathbf{P}} \left( \, #1 \, \right) }

\newcommand{\px}[1]{{\mathbf{P}_x}\!\left( \, #1 \, \right) }

\newcommand{\e}{{\mathbf{E}} }

\newcommand{\lk}{«}

\newcommand{\pk}{»}

\newcommand{\lm}{(\lambda,\mu)}

\newcommand{\sk}[1]{\left[ #1 \right]}

\newcommand{\skk}[1]{\left\{ #1 \right\}}

\newcommand{\lef}{\left(}

\newcommand{\rig}{\right)}

\newcommand{\skl}[1]{\left\langle #1 \right\rangle}



\renewcommand{\le}{\leqslant}

\renewcommand{\ge}{\geqslant}



 \def\lms{\mathop{\overline{\lim}}\limits}



\def\lmi{\mathop{\underline{\lim}}\limits}



\def\v{\varepsilon}



\def\d{\delta}

\def\a{\alpha}

\def\b{\beta}

\def\g{\gamma}

\def\l{\lambda}

\def\m{\mu}

\def\({\left(}

\def\){\right)}



\def\({\left(}

\def\){\right)}



\def\[{\left[}

\def\]{\right]}





\def\|{\left|}

\def\|{\right|}







\def\SL{\left\{}

\def\SP{\right\}}



\def\dd{{\boldsymbol{\delta}}}

\def\aa{{\boldsymbol{\alpha}}}

\def\bb{{\boldsymbol{\beta}}}

\def\gg{{\boldsymbol{\gamma}}}

\def\ll{{\boldsymbol{lambda}}}

\def\mm{{\boldsymbol{\mu}}}

\def\0{{\boldsymbol{0}}}





\def\n{\nu}

\def\t{\tau}

\def\x{\xi}

\def\z{\zeta}

\def\zz{{\boldsymbol{\zeta}}}

\def\e{\eta}

\def\h{\theta}

\def\ll{{\boldsymbol{\lambda}}}

\def\mm{{\boldsymbol{\mu}}}

\def\xx{{\boldsymbol{\xi}}}

\def\ppsi{{\boldsymbol{\psi}}}

\def\LL{{\boldsymbol{\Lambda}}}







\def\DD{Д\ о\ к\ а\ з\ а\ т\ е\ л\ ь\ с\ т\ в\ о}

\def\D{\mathbb D}

\def\C{\mathbb C}

\def\R{\mathbb R}

\def\Z{\mathbb Z}

\def\V{\mathbb V}





\def\r{\rho}

\def\x{\xi}



\pagestyle{empty} 



%    \raisebox{-1pt}[0pt][0pt]{\includegraphics[width=0.02\linewidth]{3.png}}

\begin{document} 
\begin{center} {\Large Билет №1} \end{center} 

\raisebox{-1pt}[0pt][0pt]{\includegraphics[width=0.02\linewidth]{3.png}} Опр. (гамма распределения). Доказательство, что это действительно распределение. \\

\raisebox{-1pt}[0pt][0pt]{\includegraphics[width=0.02\linewidth]{3.png}} Замечание (о вычисление  математического ожидания для преобразований случайных величин (одномерных и многомерных преобразований)). \\

\raisebox{-1pt}[0pt][0pt]{\includegraphics[width=0.02\linewidth]{4.png}} Следствие (об идеальной зависимости). Доказательство. \\

\raisebox{-1pt}[0pt][0pt]{\includegraphics[width=0.02\linewidth]{4.png}} Примеры базовых копул. \\

\raisebox{-1pt}[0pt][0pt]{\includegraphics[width=0.02\linewidth]{5.png}} Теорема (об инвариантности копулы при строго возрастающем преобразовании). Доказательство. \\

\raisebox{-1pt}[0pt][0pt]{\includegraphics[width=0.02\linewidth]{5.png}} Теорема Слуцкого. Доказательство. \\

\begin{center} {\Large Билет №2} \end{center} 

\raisebox{-1pt}[0pt][0pt]{\includegraphics[width=0.02\linewidth]{3.png}} Опр. (функции распределения). Примеры. \\

\raisebox{-1pt}[0pt][0pt]{\includegraphics[width=0.02\linewidth]{3.png}} Опр. (гауссовской копулы). \\

\raisebox{-1pt}[0pt][0pt]{\includegraphics[width=0.02\linewidth]{4.png}} Свойства дисперсии с доказательством (дисперсия суммы независимых с.в., оптимизационная задача). \\ 

\raisebox{-1pt}[0pt][0pt]{\includegraphics[width=0.02\linewidth]{4.png}} Свойства дисперсии с доказательством (альтернативный способ вычисления, критерий вырожденности, линейные преобр. одной случайной величины ). \\ 

\raisebox{-1pt}[0pt][0pt]{\includegraphics[width=0.02\linewidth]{5.png}}   Теорема (о квантилях и линейном преобазование случайных величин, обобщенная обратная функция). Доказательство. \\

\raisebox{-1pt}[0pt][0pt]{\includegraphics[width=0.02\linewidth]{5.png}} Теорема (закон больших чисел Колмогорова). Доказательство достаточности при 4-ом моменте. \\

\begin{center} {\Large Билет №3} \end{center} 

\raisebox{-1pt}[0pt][0pt]{\includegraphics[width=0.02\linewidth]{3.png}} Опр. (копулы). \\

\raisebox{-1pt}[0pt][0pt]{\includegraphics[width=0.02\linewidth]{3.png}} Теорема (закон больших чисел Колмогорова) (без док-ва). \\

\raisebox{-1pt}[0pt][0pt]{\includegraphics[width=0.02\linewidth]{4.png}} Свойство  математического ожидания для независимых случайных величин. Доказательство. \\

\raisebox{-1pt}[0pt][0pt]{\includegraphics[width=0.02\linewidth]{4.png}} Теорема (неравенство Йенсена). Доказательство. \\

\raisebox{-1pt}[0pt][0pt]{\includegraphics[width=0.02\linewidth]{5.png}} Свойства характеристических функций с доказательством (значение в нуле, линейное преобразование, сумма независимых, гладкость в нуле). \\

\raisebox{-1pt}[0pt][0pt]{\includegraphics[width=0.02\linewidth]{5.png}} Теорема (формула обращения). Доказательство. \\

\begin{center} {\Large Билет №4} \end{center} 

\raisebox{-1pt}[0pt][0pt]{\includegraphics[width=0.02\linewidth]{3.png}} Опр. ($k-$ого момента, $k$-ого центрального момента). Формулы для вычисления у дискретного и а.н.р. \\

\raisebox{-1pt}[0pt][0pt]{\includegraphics[width=0.02\linewidth]{3.png}}   Опр. (квантили для непрерывной функции распределения). \\

\raisebox{-1pt}[0pt][0pt]{\includegraphics[width=0.02\linewidth]{4.png}} Теорема (свертка для дискретных).  Доказательство. \\

\raisebox{-1pt}[0pt][0pt]{\includegraphics[width=0.02\linewidth]{4.png}} Лемма (о единственности предела для математического ожидания от простых). Доказательство. \\ 

\raisebox{-1pt}[0pt][0pt]{\includegraphics[width=0.02\linewidth]{5.png}} Теорема (о правой границе неравенства Frechet-Hoeffding).  Доказательство. \\

\raisebox{-1pt}[0pt][0pt]{\includegraphics[width=0.02\linewidth]{5.png}} Вычисление интеграла от плотности нормального распределения.  Свойства нормального распределения с доказательством (лин. преобр., равенства для $\Phi_{0,1},$ правило трех сигм). \\

\begin{center} {\Large Билет №5} \end{center} 

\raisebox{-1pt}[0pt][0pt]{\includegraphics[width=0.02\linewidth]{3.png}} Опр. (сходимости по вероятности). \\

%    \raisebox{-1pt}[0pt][0pt]{\includegraphics[width=0.02\linewidth]{3.png}}

\raisebox{-1pt}[0pt][0pt]{\includegraphics[width=0.02\linewidth]{4.png}} Следствие (о независимости и корреляции для нормального вектора). Доказательство. \\

\raisebox{-1pt}[0pt][0pt]{\includegraphics[width=0.02\linewidth]{4.png}} Классическое неравенство Чебышёва. Доказательство. \\

\raisebox{-1pt}[0pt][0pt]{\includegraphics[width=0.02\linewidth]{5.png}} Теорема (закон больших чисел Колмогорова). Доказательство достаточности при 4-ом моменте. \\

\raisebox{-1pt}[0pt][0pt]{\includegraphics[width=0.02\linewidth]{5.png}} Теорема (о правой границе неравенства Frechet-Hoeffding).  Доказательство. \\

\begin{center} {\Large Билет №6} \end{center} 

\raisebox{-1pt}[0pt][0pt]{\includegraphics[width=0.02\linewidth]{3.png}} Опр. (сходимости почти наверное). \\

\raisebox{-1pt}[0pt][0pt]{\includegraphics[width=0.02\linewidth]{3.png}} Опр. (распределения случайной величины). Примеры распределений, как вероятностных мер. \\

\raisebox{-1pt}[0pt][0pt]{\includegraphics[width=0.02\linewidth]{4.png}} Теорема (формула полной вероятности). Доказательство. \\

\raisebox{-1pt}[0pt][0pt]{\includegraphics[width=0.02\linewidth]{4.png}} Теорема (формула Бернулли). Доказательство. \\

\raisebox{-1pt}[0pt][0pt]{\includegraphics[width=0.02\linewidth]{5.png}}  Теорема (о квантильном преобразование). Доказательство. \\

\raisebox{-1pt}[0pt][0pt]{\includegraphics[width=0.02\linewidth]{5.png}} Свойства характеристических функций с доказательством (значение в нуле, линейное преобразование, сумма независимых, гладкость в нуле). \\

\begin{center} {\Large Билет №7} \end{center} 

\raisebox{-1pt}[0pt][0pt]{\includegraphics[width=0.02\linewidth]{3.png}} Опр. (распределения Парето). Доказательство, что это действительно распределение. Вычисление функции распределения. \\

\raisebox{-1pt}[0pt][0pt]{\includegraphics[width=0.02\linewidth]{3.png}} Аксиомы геометрической вероятностной модели. Примеры. \\

\raisebox{-1pt}[0pt][0pt]{\includegraphics[width=0.02\linewidth]{4.png}} Теорема (о плотности и линейном преобразование случайных величин). Доказательство. \\

\raisebox{-1pt}[0pt][0pt]{\includegraphics[width=0.02\linewidth]{4.png}} Следствие (об идеальной зависимости). Доказательство. \\

\raisebox{-1pt}[0pt][0pt]{\includegraphics[width=0.02\linewidth]{5.png}} Теорема (оценка точности в теореме Пуассона). Доказательство. \\

\raisebox{-1pt}[0pt][0pt]{\includegraphics[width=0.02\linewidth]{5.png}} Теорема (УМО для гауссовских векторов). Доказательство. \\

\begin{center} {\Large Билет №8} \end{center} 

\raisebox{-1pt}[0pt][0pt]{\includegraphics[width=0.02\linewidth]{3.png}}  Опр. (геометрического распределения). Доказательство, что это действительно распределение. Свойство нестарения. Пример случайных экспериментов и случайной величины с этим распределением. \\      

\raisebox{-1pt}[0pt][0pt]{\includegraphics[width=0.02\linewidth]{3.png}} Свойства функций распределения (без док-ва). \\

\raisebox{-1pt}[0pt][0pt]{\includegraphics[width=0.02\linewidth]{4.png}} Лемма (о приближение случайной величины простыми). Доказательство. \\

\raisebox{-1pt}[0pt][0pt]{\includegraphics[width=0.02\linewidth]{4.png}} Свойства дисперсии с доказательством (альтернативный способ вычисления, критерий вырожденности, линейные преобр. одной случайной величины ). \\ 

\raisebox{-1pt}[0pt][0pt]{\includegraphics[width=0.02\linewidth]{5.png}} Теорема (УМО для гауссовских векторов). Доказательство. \\

\raisebox{-1pt}[0pt][0pt]{\includegraphics[width=0.02\linewidth]{5.png}} Вычисление интеграла от плотности нормального распределения.  Свойства нормального распределения с доказательством (лин. преобр., равенства для $\Phi_{0,1},$ правило трех сигм). \\

\begin{center} {\Large Билет №9} \end{center} 

\raisebox{-1pt}[0pt][0pt]{\includegraphics[width=0.02\linewidth]{3.png}} Лемма (вычисление УМО для а.н.р.)  (без док-ва). \\

\raisebox{-1pt}[0pt][0pt]{\includegraphics[width=0.02\linewidth]{3.png}} Опр. (случайной величины). \\ 

\raisebox{-1pt}[0pt][0pt]{\includegraphics[width=0.02\linewidth]{4.png}} Следствие (о независимости и корреляции для нормального вектора). Доказательство. \\ 

\raisebox{-1pt}[0pt][0pt]{\includegraphics[width=0.02\linewidth]{4.png}} Свойства математического ожидания для  простых случайных величин с доказательством. \\

\raisebox{-1pt}[0pt][0pt]{\includegraphics[width=0.02\linewidth]{5.png}} Теорема (УМО для гауссовских векторов). Доказательство. \\

\raisebox{-1pt}[0pt][0pt]{\includegraphics[width=0.02\linewidth]{5.png}} Теорема (закон больших чисел Колмогорова). Доказательство достаточности при 4-ом моменте. \\

\begin{center} {\Large Билет №10} \end{center} 

\raisebox{-1pt}[0pt][0pt]{\includegraphics[width=0.02\linewidth]{3.png}} Теорема (неравенство Маркова). (без док-ва) \\

\raisebox{-1pt}[0pt][0pt]{\includegraphics[width=0.02\linewidth]{3.png}} Опр. (вероятностной меры). \\

\raisebox{-1pt}[0pt][0pt]{\includegraphics[width=0.02\linewidth]{4.png}} Теорема (формула полной вероятности). Доказательство. \\

\raisebox{-1pt}[0pt][0pt]{\includegraphics[width=0.02\linewidth]{4.png}} Примеры вычисления характеристических функций (вырожденное, Пуассона, нормальное). \\

\raisebox{-1pt}[0pt][0pt]{\includegraphics[width=0.02\linewidth]{5.png}} Вычисление интеграла от плотности нормального распределения.  Свойства нормального распределения с доказательством (лин. преобр., равенства для $\Phi_{0,1},$ правило трех сигм). \\

\raisebox{-1pt}[0pt][0pt]{\includegraphics[width=0.02\linewidth]{5.png}} Теорема об ортогональной проекции. Доказательство. \\

\end{document}