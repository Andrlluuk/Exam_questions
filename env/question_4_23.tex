

\documentclass[10pt]{article}

\usepackage[utf8]{inputenc}

\usepackage[english,russian]{babel}

\usepackage{amsmath}

\usepackage{amssymb}

\usepackage{amsfonts}

\usepackage{chngcntr,color}

%\usepackage[hyper]{amsbib}

\usepackage{ulem}

\usepackage[hidelinks]{hyperref}

\usepackage[mathscr]{euscript}

\usepackage{enumitem}

\usepackage{cite}

\usepackage{bbm} % We need this for pretty Indicator \mathbbm{1}.

\usepackage[left=25mm, top=20mm, right=10mm, bottom=10mm, nohead]{geometry}

    \usepackage{graphicx} % для картинок

\pagestyle{empty} 



%    \raisebox{-1pt}[0pt][0pt]{\includegraphics[width=0.02\linewidth]{3.png}}

\begin{document}



\thispagestyle{empty}

\topskip=0pt

\vspace*{\fill}

\begin{center} {\Large Билет №23 (30 баллов)} \end{center}

\raisebox{-1pt}[0pt][0pt]{\includegraphics[width=0.02\linewidth]{4.png}} Примеры вычисления дисперсии (Бернулли, биномиального и нормального). \\

\raisebox{-1pt}[0pt][0pt]{\includegraphics[width=0.02\linewidth]{4.png}} Теорема (о плотности и линейном преобразование случайных величин). Доказательство. \\

\raisebox{-1pt}[0pt][0pt]{\includegraphics[width=0.02\linewidth]{4.png}} Теорема (Шкляра). Доказательство в непрерывном случае. \\









\vspace*{\fill}

\end{document}
