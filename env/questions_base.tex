
\documentclass[10pt]{amsart}
%\usepackage[cp1251]{inputenc}
\usepackage[english,russian]{babel}
\usepackage{amsmath}
\usepackage{amssymb}
\usepackage{amsfonts}
\usepackage{chngcntr,color}
%\usepackage[hyper]{amsbib}
\usepackage{ulem}
\usepackage[hidelinks]{hyperref}
\usepackage[mathscr]{euscript}
\usepackage{enumitem}
\usepackage{cite}
\usepackage{bbm} % We need this for pretty Indicator \mathbbm{1}.
\usepackage[left=25mm, top=20mm, right=10mm, bottom=10mm, nohead, nofoot]{geometry}
    \usepackage{graphicx} % для картинок


\renewcommand\theequation{\arabic{section}.\arabic{equation}}
\newcommand{\ra}{\rightarrow}
\newcommand{\p}[1]{{\mathbf{P}} \left( \, #1 \, \right) }
\newcommand{\px}[1]{{\mathbf{P}_x}\!\left( \, #1 \, \right) }
\newcommand{\e}{{\mathbf{E}} }
\newcommand{\lk}{«}
\newcommand{\pk}{»}
\newcommand{\lm}{(\lambda,\mu)}
\newcommand{\sk}[1]{\left[ #1 \right]}
\newcommand{\skk}[1]{\left\{ #1 \right\}}
\newcommand{\lef}{\left(}
\newcommand{\rig}{\right)}
\newcommand{\skl}[1]{\left\langle #1 \right\rangle}

\renewcommand{\le}{\leqslant}
\renewcommand{\ge}{\geqslant}

 \def\lms{\mathop{\overline{\lim}}\limits}

\def\lmi{\mathop{\underline{\lim}}\limits}

\def\v{\varepsilon}

\def\d{\delta}
\def\a{\alpha}
\def\b{\beta}
\def\g{\gamma}
\def\l{\lambda}
\def\m{\mu}
\def\({\left(}
\def\){\right)}

\def\({\left(}
\def\){\right)}

\def\[{\left[}
\def\]{\right]}


\def\|{\left|}
\def\|{\right|}



\def\SL{\left\{}
\def\SP{\right\}}

\def\dd{{\boldsymbol{\delta}}}
\def\aa{{\boldsymbol{\alpha}}}
\def\bb{{\boldsymbol{\beta}}}
\def\gg{{\boldsymbol{\gamma}}}
\def\ll{{\boldsymbol{lambda}}}
\def\mm{{\boldsymbol{\mu}}}
\def\0{{\boldsymbol{0}}}


\def\n{\nu}
\def\t{\tau}
\def\x{\xi}
\def\z{\zeta}
\def\zz{{\boldsymbol{\zeta}}}
\def\e{\eta}
\def\h{\theta}
\def\ll{{\boldsymbol{\lambda}}}
\def\mm{{\boldsymbol{\mu}}}
\def\xx{{\boldsymbol{\xi}}}
\def\ppsi{{\boldsymbol{\psi}}}
\def\LL{{\boldsymbol{\Lambda}}}



\def\DD{Д\ о\ к\ а\ з\ а\ т\ е\ л\ ь\ с\ т\ в\ о}
\def\D{\mathbb D}
\def\C{\mathbb C}
\def\R{\mathbb R}
\def\Z{\mathbb Z}
\def\V{\mathbb V}


\def\r{\rho}
\def\x{\xi}


%    \raisebox{-1pt}[0pt][0pt]{\includegraphics[width=0.02\linewidth]{3.png}}
\begin{document}


\thispagestyle{empty}

\begin{center}
    {\Large Программа по теории вероятностей 2020}
\end{center}


\begin{enumerate}
\item[Глава 1.] Случайные события и их вероятности \\
\begin{enumerate}
\item[\S\, 1.1.] Элементарная теория вероятностей \\
 \raisebox{-1pt}[0pt][0pt]{\includegraphics[width=0.02\linewidth]{3.png}} Опр. (пространства элементарных исходов), примеры пространств элементарных исходов и событий. \\
 \raisebox{-1pt}[0pt][0pt]{\includegraphics[width=0.02\linewidth]{3.png}} Опр. (операций над событиями, несовместных событий). \\
\item[\S\, 1.2.] Модель дискретной вероятности \\
 \raisebox{-1pt}[0pt][0pt]{\includegraphics[width=0.02\linewidth]{3.png}} Аксиомы классической вероятностной модели. Примеры случайных экспериментов, которые одновременно могут быть описаны классической и неклассической вероятностной схемой. \\
 \raisebox{-1pt}[0pt][0pt]{\includegraphics[width=0.02\linewidth]{3.png}} Аксиомы дискретной вероятности. Примеры. \\
\item[\S\, 1.3.] Геометрические вероятности. \\
 \raisebox{-1pt}[0pt][0pt]{\includegraphics[width=0.02\linewidth]{3.png}} Аксиомы геометрической вероятностной модели. Примеры. \\
 \raisebox{-1pt}[0pt][0pt]{\includegraphics[width=0.02\linewidth]{4.png}} Пример (парадокс Бертрана). \\ 

\item[\S\, 1.4.] Аксиоматическое построение теории вероятностей \\ 
 \raisebox{-1pt}[0pt][0pt]{\includegraphics[width=0.02\linewidth]{3.png}} Опр.( сигма – алгебры). Примеры сигма-алгебр. \\
 \raisebox{-1pt}[0pt][0pt]{\includegraphics[width=0.02\linewidth]{4.png}} Свойства сигма-алгебр с доказательствами (пустое мно-во, конечное объединение, счетное пересечение). Формула двойственности. \\
 \raisebox{-1pt}[0pt][0pt]{\includegraphics[width=0.02\linewidth]{3.png}} Опр. (борелевской сигма-алгебра). \\
 \raisebox{-1pt}[0pt][0pt]{\includegraphics[width=0.02\linewidth]{3.png}} Опр. (вероятностной меры). \\
 \raisebox{-1pt}[0pt][0pt]{\includegraphics[width=0.02\linewidth]{4.png}} Свойства вероятностной меры с доказательствами (вероятность пустого мн-ва, дизъюнктного объединения, дополнения, объединения двух мн-в, монотонность). \\
 \raisebox{-1pt}[0pt][0pt]{\includegraphics[width=0.02\linewidth]{5.png}} Свойства  вероятностной меры с доказательством (вероятность объединения счетного набора, непрерывность вер. меры, формула включения/исключения). \\
 \raisebox{-1pt}[0pt][0pt]{\includegraphics[width=0.02\linewidth]{3.png}} Опр. (вероятностного пространства). Примеры. \\

\item[\S\, 1.5.] Условная вероятность, независимость \\
 \raisebox{-1pt}[0pt][0pt]{\includegraphics[width=0.02\linewidth]{3.png}} Опр. (условной вероятности).  Свойство (о перемножении вероятностей). \\ 
 \raisebox{-1pt}[0pt][0pt]{\includegraphics[width=0.02\linewidth]{3.png}} Опр. (двух независимых событий). \\
 \raisebox{-1pt}[0pt][0pt]{\includegraphics[width=0.02\linewidth]{4.png}} Свойства независимых событий с доказательством (несовместность, условная вероятность, \\ теоретико-множественные операции). \\
 \raisebox{-1pt}[0pt][0pt]{\includegraphics[width=0.02\linewidth]{3.png}} Опр. (событий, независимых в совокупности). \\
 \raisebox{-1pt}[0pt][0pt]{\includegraphics[width=0.02\linewidth]{4.png}}  Пример (Бернштейна). \\


    \item[\S\, 1.6.] Схема Бернулли \\
 \raisebox{-1pt}[0pt][0pt]{\includegraphics[width=0.02\linewidth]{3.png}} Опр. (Схемы Бернулли). Примеры экспериментов со схемой Бернулли. \\
 \raisebox{-1pt}[0pt][0pt]{\includegraphics[width=0.02\linewidth]{3.png}} Теорема (формула Бернулли). (без док-ва). \\
 \raisebox{-1pt}[0pt][0pt]{\includegraphics[width=0.02\linewidth]{4.png}} Теорема (формула Бернулли). Доказательство. \\
 \raisebox{-1pt}[0pt][0pt]{\includegraphics[width=0.02\linewidth]{4.png}} Теорема Пуассона для схемы Бернулли. Доказательство. \\
 \raisebox{-1pt}[0pt][0pt]{\includegraphics[width=0.02\linewidth]{4.png}} Теорема (номер первого успешного испытания в схеме Бернулли). Доказательство. \\
 \raisebox{-1pt}[0pt][0pt]{\includegraphics[width=0.02\linewidth]{5.png}} Теорема (полиномиальная схема). Доказательство. \\

    
    
\item[\S\, 1.7.] Формул полной вероятности \\
 \raisebox{-1pt}[0pt][0pt]{\includegraphics[width=0.02\linewidth]{3.png}} Опр. (полной группы событий). \\
 \raisebox{-1pt}[0pt][0pt]{\includegraphics[width=0.02\linewidth]{4.png}} Теорема (формула полной вероятности). Доказательство. \\
 \raisebox{-1pt}[0pt][0pt]{\includegraphics[width=0.02\linewidth]{4.png}} Пример (задача о разорении для двух игроков при помощи ФПВ). \\ 
 \raisebox{-1pt}[0pt][0pt]{\includegraphics[width=0.02\linewidth]{4.png}} Теорема (формула Байеса). Доказательство. \\
Примеры случайных экспериментов, описываемых с помощью ФПВ и формулы Байеса. \\
\end{enumerate}
\item[Глава 2.]  Случайные величины и их распределения \\
    
\begin{enumerate}
\item[\S\, 2.1.]  Случайные величины \\
 \raisebox{-1pt}[0pt][0pt]{\includegraphics[width=0.02\linewidth]{3.png}} Опр. (случайной величины). \\ 
 \raisebox{-1pt}[0pt][0pt]{\includegraphics[width=0.02\linewidth]{4.png}} Примеры вероятностных пространств и функций, которые являются или не являются случайными величинами с доказательством. \\

    
\item[\S\, 2.2.] Распределения случайных величин \\
 \raisebox{-1pt}[0pt][0pt]{\includegraphics[width=0.02\linewidth]{3.png}} Опр. (распределения случайной величины). Примеры распределений, как вероятностных мер. \\
 \raisebox{-1pt}[0pt][0pt]{\includegraphics[width=0.02\linewidth]{3.png}} Опр. (дискретного распределения). Примеры. \\
 \raisebox{-1pt}[0pt][0pt]{\includegraphics[width=0.02\linewidth]{3.png}} Опр. (абсолютно непрерывного распределения). Примеры. \\
 \raisebox{-1pt}[0pt][0pt]{\includegraphics[width=0.02\linewidth]{4.png}} Теорема (о плотности). Доказательство. \\
 \raisebox{-1pt}[0pt][0pt]{\includegraphics[width=0.02\linewidth]{3.png}} Опр. (сингулярного распределения). Примеры. \\
 \raisebox{-1pt}[0pt][0pt]{\includegraphics[width=0.02\linewidth]{3.png}} Опр. (смешанного распределения). Примеры. \\


\item[\S\, 2.3.] Функция распределения \\
    
 \raisebox{-1pt}[0pt][0pt]{\includegraphics[width=0.02\linewidth]{3.png}} Опр. (функции распределения). Примеры. \\
 \raisebox{-1pt}[0pt][0pt]{\includegraphics[width=0.02\linewidth]{3.png}} Свойства функций распределения (без док-ва). \\
 \raisebox{-1pt}[0pt][0pt]{\includegraphics[width=0.02\linewidth]{4.png}} Свойства функций распределения с доказательствами. \\
Теорема (о классе функций распределения) (без док-ва)\\


\item[\S\, 2.4.] Примеры распределений \\
    
 \raisebox{-1pt}[0pt][0pt]{\includegraphics[width=0.02\linewidth]{3.png}} Опр. (вырожденного распределения). Пример случайных экспериментов и случайной величины с этим распределением. \\
 \raisebox{-1pt}[0pt][0pt]{\includegraphics[width=0.02\linewidth]{3.png}} Опр. (распределения Бернулли).  Пример случайных экспериментов и случайной величины с этим распределением. \\
 \raisebox{-1pt}[0pt][0pt]{\includegraphics[width=0.02\linewidth]{3.png}}  Опр. (биномиального распределения). Доказательство, что это действительно распределение. Пример случайных экспериментов и случайной величины с этим распределением. \\     
 \raisebox{-1pt}[0pt][0pt]{\includegraphics[width=0.02\linewidth]{3.png}}  Опр. (геометрического распределения). Доказательство, что это действительно распределение. Свойство нестарения. Пример случайных экспериментов и случайной величины с этим распределением. \\      
 \raisebox{-1pt}[0pt][0pt]{\includegraphics[width=0.02\linewidth]{3.png}}  Опр. (распределения Пуассона). Доказательство, что это действительно распределение. \\
 \raisebox{-1pt}[0pt][0pt]{\includegraphics[width=0.02\linewidth]{3.png}} Опр. (гипергеометрического распределения). Доказательство, что это действительно распределение. Пример случайных экспериментов и случайной величины с этим распределением. \\      
 \raisebox{-1pt}[0pt][0pt]{\includegraphics[width=0.02\linewidth]{3.png}}  Опр. (равномерного распределения). Доказательство, что это действительно распределение. Вычисление функции распределения. Пример случайных экспериментов и случайной величины с этим распределением. \\
 \raisebox{-1pt}[0pt][0pt]{\includegraphics[width=0.02\linewidth]{3.png}}  Опр. (показательного распределения). Доказательство, что это действительно распределение. Свойство нестарения. Вычисление функции распределения. \\  
 \raisebox{-1pt}[0pt][0pt]{\includegraphics[width=0.02\linewidth]{3.png}}  Опр. (нормального (гауссовского) распределения). Свойство линейных преобразований с доказательством. \\
 \raisebox{-1pt}[0pt][0pt]{\includegraphics[width=0.02\linewidth]{5.png}} Вычисление интеграла от плотности нормального распределения.  Свойства нормального распределения с доказательством (лин. преобр., равенства для $\Phi_{0,1},$ правило трех сигм). \\
 \raisebox{-1pt}[0pt][0pt]{\includegraphics[width=0.02\linewidth]{3.png}} Опр. (гамма распределения). Доказательство, что это действительно распределение. \\
 \raisebox{-1pt}[0pt][0pt]{\includegraphics[width=0.02\linewidth]{3.png}} Опр. (распределения Коши). Доказательство, что это действительно распределение. Вычисление функции распределения. \\
 \raisebox{-1pt}[0pt][0pt]{\includegraphics[width=0.02\linewidth]{3.png}} Опр. (распределения Парето). Доказательство, что это действительно распределение. Вычисление функции распределения. \\
 \raisebox{-1pt}[0pt][0pt]{\includegraphics[width=0.02\linewidth]{3.png}} Опр. (логнормального распределения). Вычисление плотности. \\
 \raisebox{-1pt}[0pt][0pt]{\includegraphics[width=0.02\linewidth]{4.png}}   Пример сингулярного распределения (лестница Кантора). \\
 \raisebox{-1pt}[0pt][0pt]{\includegraphics[width=0.02\linewidth]{3.png}}  Опр. (смеси распределений). Пример задания смеси двойной рандомизацией. \\

\item[\S\, 2.5.] Преобразования случайных величин \\
Замечание (об измеримости преобразования случайной величины). \\
 \raisebox{-1pt}[0pt][0pt]{\includegraphics[width=0.02\linewidth]{4.png}} Теорема (о плотности и линейном преобразование случайных величин). Доказательство. \\
 \raisebox{-1pt}[0pt][0pt]{\includegraphics[width=0.02\linewidth]{3.png}}   Опр. (квантили для непрерывной функции распределения). \\
 \raisebox{-1pt}[0pt][0pt]{\includegraphics[width=0.02\linewidth]{3.png}}   Опр. (квантили в общей случае). \\
Опр. (медианы).   Опр. (моды). \\
 \raisebox{-1pt}[0pt][0pt]{\includegraphics[width=0.02\linewidth]{5.png}}   Теорема (о квантилях и линейном преобазование случайных величин, обобщенная обратная функция). Доказательство. \\
 \raisebox{-1pt}[0pt][0pt]{\includegraphics[width=0.02\linewidth]{5.png}}  Теорема (о квантильном преобразование). Доказательство. \\

    
\item[\S\, 2.6.] Многомерные распределения \\
 \raisebox{-1pt}[0pt][0pt]{\includegraphics[width=0.02\linewidth]{3.png}} Опр. (случайного вектора). \\
 \raisebox{-1pt}[0pt][0pt]{\includegraphics[width=0.02\linewidth]{3.png}} Опр. (совместного распределения и  совместной функции распределения). \\
 \raisebox{-1pt}[0pt][0pt]{\includegraphics[width=0.02\linewidth]{4.png}} Свойства совместной функции распределения. Доказательство. \\
 \raisebox{-1pt}[0pt][0pt]{\includegraphics[width=0.02\linewidth]{3.png}} Опр. (дискретного многомерного распределения). Свойства. Примеры. \\
 \raisebox{-1pt}[0pt][0pt]{\includegraphics[width=0.02\linewidth]{3.png}} Опр. (многомерного абсолютно непрерывного распределения). \\
 \raisebox{-1pt}[0pt][0pt]{\includegraphics[width=0.02\linewidth]{4.png}} Нахождение маргинальных плотностей по многомерной плотности. \\
 \raisebox{-1pt}[0pt][0pt]{\includegraphics[width=0.02\linewidth]{3.png}} Опр. (многомерного равномерного распределения). \\
 \raisebox{-1pt}[0pt][0pt]{\includegraphics[width=0.02\linewidth]{3.png}} Опр. (многомерного нормального распределения). Вид плотности для многомерного стандартного нормального вектора. \\ 
Опр. (многомерного сингулярного распределения). Пример. \\
Опр. (распределения Дирихле). \\
 \raisebox{-1pt}[0pt][0pt]{\includegraphics[width=0.02\linewidth]{3.png}} Опр. (независимых случайных величин). \\
 \raisebox{-1pt}[0pt][0pt]{\includegraphics[width=0.02\linewidth]{3.png}} Теорема (об эквивалентных определениях независимости) (без док-ва). \\
 \raisebox{-1pt}[0pt][0pt]{\includegraphics[width=0.02\linewidth]{5.png}} Теорема (об эквивалентных определениях независимости). Доказательство. \\
 \raisebox{-1pt}[0pt][0pt]{\includegraphics[width=0.02\linewidth]{4.png}} Теорема (о сохранение независимости при преобразованиях). Доказательство. \\ 
 \raisebox{-1pt}[0pt][0pt]{\includegraphics[width=0.02\linewidth]{4.png}} Теорема (свертка для дискретных). Доказательство. \\
 \raisebox{-1pt}[0pt][0pt]{\includegraphics[width=0.02\linewidth]{4.png}} Теорема (свертка для дискретных).  Доказательство. \\
\end{enumerate}
    
    
\item[Глава 3.] Числовые характеристики распределений \\
    
\begin{enumerate}
\item[\S\, 3.1.] Интеграл по вероятностной мере. Математическое ожидание. \\
 \raisebox{-1pt}[0pt][0pt]{\includegraphics[width=0.02\linewidth]{3.png}} Опр. (простой случайной величины). \\
 \raisebox{-1pt}[0pt][0pt]{\includegraphics[width=0.02\linewidth]{3.png}} Опр. (математического ожидания для простой случайной величины). \\
 \raisebox{-1pt}[0pt][0pt]{\includegraphics[width=0.02\linewidth]{4.png}} Свойства математического ожидания для  простых случайных величин с доказательством. \\
 \raisebox{-1pt}[0pt][0pt]{\includegraphics[width=0.02\linewidth]{3.png}} Опр. (математического ожидания для простой случайной величины по событию). \\
 \raisebox{-1pt}[0pt][0pt]{\includegraphics[width=0.02\linewidth]{4.png}} Лемма (о приближение случайной величины простыми). Доказательство. \\
 \raisebox{-1pt}[0pt][0pt]{\includegraphics[width=0.02\linewidth]{4.png}} Лемма (о единственности предела для математического ожидания от простых). Доказательство. \\ 
 \raisebox{-1pt}[0pt][0pt]{\includegraphics[width=0.02\linewidth]{3.png}} Опр. (математического ожидания). \\
 \raisebox{-1pt}[0pt][0pt]{\includegraphics[width=0.02\linewidth]{4.png}} Основные свойства математического ожидания с доказательством (Однородность, монотонность, нер-во треугольника, аддитивность). \\
 \raisebox{-1pt}[0pt][0pt]{\includegraphics[width=0.02\linewidth]{5.png}}  Свойство счетной аддитивности математического ожидания. Доказательство. \\
 \raisebox{-1pt}[0pt][0pt]{\includegraphics[width=0.02\linewidth]{3.png}} Свойство  математического ожидания для независимых случайных величин (без док-ва). \\
 \raisebox{-1pt}[0pt][0pt]{\includegraphics[width=0.02\linewidth]{4.png}} Свойство  математического ожидания для независимых случайных величин. Доказательство. \\
 \raisebox{-1pt}[0pt][0pt]{\includegraphics[width=0.02\linewidth]{3.png}} Замечание (о вычисление  математического ожидания для дискретных, для а.н.р). \\
 \raisebox{-1pt}[0pt][0pt]{\includegraphics[width=0.02\linewidth]{3.png}} Замечание (о вычисление  математического ожидания для преобразований случайных величин (одномерных и многомерных преобразований)). \\
 \raisebox{-1pt}[0pt][0pt]{\includegraphics[width=0.02\linewidth]{5.png}} Теорема (о свертке для произвольных распределений). Доказательство. Следствие об а.н.р. суммы. \\ 
 \raisebox{-1pt}[0pt][0pt]{\includegraphics[width=0.02\linewidth]{4.png}} Примеры вычисления математического ожидания (Бернулли, биномиальное (двумя способами,нормальное). \\


\item[\S\, 3.2.] Моменты высшего порядка \\
 \raisebox{-1pt}[0pt][0pt]{\includegraphics[width=0.02\linewidth]{3.png}} Опр. ($k-$ого момента, $k$-ого центрального момента). Формулы для вычисления у дискретного и а.н.р. \\
 \raisebox{-1pt}[0pt][0pt]{\includegraphics[width=0.02\linewidth]{4.png}} Теорема (о существование математического ожидания меньших порядков). Доказательство. \\


\item[\S\, 3.3.] Моментные Неравенства \\
 \raisebox{-1pt}[0pt][0pt]{\includegraphics[width=0.02\linewidth]{3.png}} Теорема (неравенство Маркова). (без док-ва) \\
 \raisebox{-1pt}[0pt][0pt]{\includegraphics[width=0.02\linewidth]{4.png}} Теорема (неравенство Маркова). Доказательство. \\
 \raisebox{-1pt}[0pt][0pt]{\includegraphics[width=0.02\linewidth]{4.png}} Следствие из неравенства Маркова о распределение неотрицательной с.в. с нулевым МО. Доказательство. \\
 \raisebox{-1pt}[0pt][0pt]{\includegraphics[width=0.02\linewidth]{3.png}} Теорема (обобщенное неравенство Чебышёва) (без док-ва). \\
 \raisebox{-1pt}[0pt][0pt]{\includegraphics[width=0.02\linewidth]{4.png}} Теорема (обобщенное неравенство Чебышёва). Доказательство. \\
 \raisebox{-1pt}[0pt][0pt]{\includegraphics[width=0.02\linewidth]{4.png}} Теорема (неравенство Коши-Буняковского). Доказательство. \\
 \raisebox{-1pt}[0pt][0pt]{\includegraphics[width=0.02\linewidth]{3.png}} Теорема (неравенство Йенсена) (без док-ва). \\
 \raisebox{-1pt}[0pt][0pt]{\includegraphics[width=0.02\linewidth]{4.png}} Теорема (неравенство Йенсена). Доказательство. \\

\item[\S\, 3.4.] Дисперсия \\
 \raisebox{-1pt}[0pt][0pt]{\includegraphics[width=0.02\linewidth]{3.png}} Опр. (дисперсии, стандартного отклонения). \\ 
 \raisebox{-1pt}[0pt][0pt]{\includegraphics[width=0.02\linewidth]{4.png}} Свойства дисперсии с доказательством (альтернативный способ вычисления, критерий вырожденности, линейные преобр. одной случайной величины ). \\ 
 \raisebox{-1pt}[0pt][0pt]{\includegraphics[width=0.02\linewidth]{4.png}} Свойства дисперсии с доказательством (дисперсия суммы независимых с.в., оптимизационная задача). \\ 
 \raisebox{-1pt}[0pt][0pt]{\includegraphics[width=0.02\linewidth]{4.png}} Классическое неравенство Чебышёва. Доказательство. \\
 \raisebox{-1pt}[0pt][0pt]{\includegraphics[width=0.02\linewidth]{4.png}} Примеры вычисления дисперсии (Бернулли, биномиального и нормального). \\
 
        
        
\item[\S\, 3.5.] Коэффициент корреляции \\
 \raisebox{-1pt}[0pt][0pt]{\includegraphics[width=0.02\linewidth]{3.png}} Опр. (ковариации двух случайных величин). \\
 \raisebox{-1pt}[0pt][0pt]{\includegraphics[width=0.02\linewidth]{3.png}} Свойства ковариации (без док-ва). \\
 \raisebox{-1pt}[0pt][0pt]{\includegraphics[width=0.02\linewidth]{4.png}} Свойства ковариации с доказательством. \\
 \raisebox{-1pt}[0pt][0pt]{\includegraphics[width=0.02\linewidth]{3.png}} Опр. (коэффициента корреляции). \\
 \raisebox{-1pt}[0pt][0pt]{\includegraphics[width=0.02\linewidth]{3.png}} Свойства коэффициента корреляции (без док-ва). \\
 \raisebox{-1pt}[0pt][0pt]{\includegraphics[width=0.02\linewidth]{5.png}} Свойства коэффициента корреляции с доказательством. \\



       
        

\item[\S\, 3.6.] Матрица ковариации \\
 \raisebox{-1pt}[0pt][0pt]{\includegraphics[width=0.02\linewidth]{3.png}} Опр. (математического ожидания для случайного вектора и случайной матрицы). \\
 \raisebox{-1pt}[0pt][0pt]{\includegraphics[width=0.02\linewidth]{4.png}} Свойства многомерного математического ожидания (линейность, произведение независимых матриц). Доказательство. \\ 
 \raisebox{-1pt}[0pt][0pt]{\includegraphics[width=0.02\linewidth]{3.png}} Опр. (матрицы ковариации случайного вектора). \\
 \raisebox{-1pt}[0pt][0pt]{\includegraphics[width=0.02\linewidth]{4.png}} Свойства матрицы ковариации (при линейном преобразование, для суммы независимых случайных векторов). Доказательство. \\


\item[\S\, 3.7.] Многомерное нормальное распределение \\

 \raisebox{-1pt}[0pt][0pt]{\includegraphics[width=0.02\linewidth]{3.png}} Теорема (о линейном преобразование для нормального вектора) (без док-ва). \\
 \raisebox{-1pt}[0pt][0pt]{\includegraphics[width=0.02\linewidth]{5.png}} Теорема (о линейном преобразование для нормального вектора). Доказательство. \\
 \raisebox{-1pt}[0pt][0pt]{\includegraphics[width=0.02\linewidth]{3.png}} Следствие (о независимости и корреляции для нормального вектора) (без док-ва). \\ 
 \raisebox{-1pt}[0pt][0pt]{\includegraphics[width=0.02\linewidth]{4.png}} Следствие (о независимости и корреляции для нормального вектора). Доказательство. \\ 
 \raisebox{-1pt}[0pt][0pt]{\includegraphics[width=0.02\linewidth]{3.png}} Следствие (о независимости и  ортогональном преобразование нормального вектора) (без док-ва). \\ 
 \raisebox{-1pt}[0pt][0pt]{\includegraphics[width=0.02\linewidth]{4.png}} Следствие (о независимости и  ортогональном преобразование нормального вектора). Доказательство. \\ 
 \raisebox{-1pt}[0pt][0pt]{\includegraphics[width=0.02\linewidth]{4.png}} Контрпример не нормального вектора с нормальными одномерными компонентами. Доказательство. \\ 

\item[\S\, 3.8.] Копулы \\
        
 \raisebox{-1pt}[0pt][0pt]{\includegraphics[width=0.02\linewidth]{3.png}} Опр. (копулы). \\
 \raisebox{-1pt}[0pt][0pt]{\includegraphics[width=0.02\linewidth]{4.png}} Теорема (Шкляра). Доказательство в непрерывном случае. \\
 \raisebox{-1pt}[0pt][0pt]{\includegraphics[width=0.02\linewidth]{4.png}} Примеры базовых копул. \\
 \raisebox{-1pt}[0pt][0pt]{\includegraphics[width=0.02\linewidth]{4.png}} Теорема (неравенства Frechet-Hoeffding). Доказательство. \\
 \raisebox{-1pt}[0pt][0pt]{\includegraphics[width=0.02\linewidth]{3.png}} Опр (носителя случайной величины). \\
Опр. (неубывающего множества). \\
 \raisebox{-1pt}[0pt][0pt]{\includegraphics[width=0.02\linewidth]{5.png}} Теорема (о правой границе неравенства Frechet-Hoeffding).  Доказательство. \\
 \raisebox{-1pt}[0pt][0pt]{\includegraphics[width=0.02\linewidth]{4.png}} Следствие (об идеальной зависимости). Доказательство. \\
 \raisebox{-1pt}[0pt][0pt]{\includegraphics[width=0.02\linewidth]{5.png}} Теорема (об инвариантности копулы при строго возрастающем преобразовании). Доказательство. \\
 \raisebox{-1pt}[0pt][0pt]{\includegraphics[width=0.02\linewidth]{3.png}} Опр. (коэффициента корреляции Спирмена). \\
 \raisebox{-1pt}[0pt][0pt]{\includegraphics[width=0.02\linewidth]{3.png}} Опр. (коэффициента корреляции Кендалла). \\
 \raisebox{-1pt}[0pt][0pt]{\includegraphics[width=0.02\linewidth]{5.png}} Свойства коэффициентов корреляции Спирмена и Кендалла. Доказательство. \\ 
 \raisebox{-1pt}[0pt][0pt]{\includegraphics[width=0.02\linewidth]{3.png}} Опр. (гауссовской копулы). \\
 \raisebox{-1pt}[0pt][0pt]{\includegraphics[width=0.02\linewidth]{3.png}} Опр. (коэффициентов экстремальной зависимости). \\
 \raisebox{-1pt}[0pt][0pt]{\includegraphics[width=0.02\linewidth]{4.png}} Лемма (о коэффициентах экстремальной зависимости в непрерывном случае). Доказательство. \\
\end{enumerate}
    
    
\item[Глава 4.] Сходимость случайных величин и распределений. Предельные теоремы \\
\begin{enumerate}
\item[\S\, 4.1.] Сходимость последовательностей случайных величин \\
 \raisebox{-1pt}[0pt][0pt]{\includegraphics[width=0.02\linewidth]{5.png}} Теорема (Бореля-Кантелли). Доказательство. \\
 \raisebox{-1pt}[0pt][0pt]{\includegraphics[width=0.02\linewidth]{3.png}} Опр. (сходимости почти наверное). \\
 \raisebox{-1pt}[0pt][0pt]{\includegraphics[width=0.02\linewidth]{3.png}} Опр. (сходимости по вероятности). \\
 \raisebox{-1pt}[0pt][0pt]{\includegraphics[width=0.02\linewidth]{3.png}} Опр. (слабой сходимости). \\
Замечание (о равномерной сходимости, если ф.р. непрерывна). Доказательство. \\
 \raisebox{-1pt}[0pt][0pt]{\includegraphics[width=0.02\linewidth]{3.png}} Опр. (сходимости в среднеквадратическом). \\
 \raisebox{-1pt}[0pt][0pt]{\includegraphics[width=0.02\linewidth]{4.png}} Лемма (критерий сходимости п.н.). Доказательство. \\
 \raisebox{-1pt}[0pt][0pt]{\includegraphics[width=0.02\linewidth]{3.png}} Теорема (п.н. vs по вероятности)  (без док-ва). \\
 \raisebox{-1pt}[0pt][0pt]{\includegraphics[width=0.02\linewidth]{5.png}} Теорема (п.н. vs по вероятности). Доказательство. \\
 \raisebox{-1pt}[0pt][0pt]{\includegraphics[width=0.02\linewidth]{3.png}} Теорема (по вероятности vs слабая) (без док-ва). \\  
 \raisebox{-1pt}[0pt][0pt]{\includegraphics[width=0.02\linewidth]{5.png}} Теорема (по вероятности vs слабая). Доказательство. \\  
 \raisebox{-1pt}[0pt][0pt]{\includegraphics[width=0.02\linewidth]{3.png}} Теорема (с.к.с. vs p vs п.н.) (без док-ва). \\ 
 \raisebox{-1pt}[0pt][0pt]{\includegraphics[width=0.02\linewidth]{4.png}} Теорема (с.к.с. vs p vs п.н.). Доказательство. \\ 

        
        
\item[\S\, 4.2.] Свойства сходимостей \\
        
 \raisebox{-1pt}[0pt][0pt]{\includegraphics[width=0.02\linewidth]{3.png}} Теорема (критерий сходимости по распределению) (без док-ва). \\
 \raisebox{-1pt}[0pt][0pt]{\includegraphics[width=0.02\linewidth]{4.png}} Лемма  (сходимость при непрерывных преобразованиях). Доказательство. \\
 \raisebox{-1pt}[0pt][0pt]{\includegraphics[width=0.02\linewidth]{4.png}} Лемма (сходимость и арифметические операции). Доказательство. \\
 \raisebox{-1pt}[0pt][0pt]{\includegraphics[width=0.02\linewidth]{3.png}} Теорема Слуцкого (без док-ва). \\
 \raisebox{-1pt}[0pt][0pt]{\includegraphics[width=0.02\linewidth]{5.png}} Теорема Слуцкого. Доказательство. \\
 \raisebox{-1pt}[0pt][0pt]{\includegraphics[width=0.02\linewidth]{3.png}} Опр. (равномерной интегрируемости). \\
 \raisebox{-1pt}[0pt][0pt]{\includegraphics[width=0.02\linewidth]{3.png}} Теорема (критерий сходимости математических ожиданий) (без док-ва). \\
 \raisebox{-1pt}[0pt][0pt]{\includegraphics[width=0.02\linewidth]{4.png}} Теорема Лебега. Доказательство. \\

         
    
\item[\S\, 4.3.] Характеристические функции \\
 \raisebox{-1pt}[0pt][0pt]{\includegraphics[width=0.02\linewidth]{3.png}} Опр. (характеристической функции). \\
 \raisebox{-1pt}[0pt][0pt]{\includegraphics[width=0.02\linewidth]{3.png}} Замечание (о вычисление и существование х.ф.). \\
 \raisebox{-1pt}[0pt][0pt]{\includegraphics[width=0.02\linewidth]{3.png}} Свойства характеристических функций (значение в нуле, линейное преобразование, сумма независимых, гладкость в нуле) (без док-ва). \\
 \raisebox{-1pt}[0pt][0pt]{\includegraphics[width=0.02\linewidth]{5.png}} Свойства характеристических функций с доказательством (значение в нуле, линейное преобразование, сумма независимых, гладкость в нуле). \\
 \raisebox{-1pt}[0pt][0pt]{\includegraphics[width=0.02\linewidth]{4.png}} Примеры вычисления характеристических функций (вырожденное, Пуассона, нормальное). \\
 \raisebox{-1pt}[0pt][0pt]{\includegraphics[width=0.02\linewidth]{5.png}} Теорема (формула обращения). Доказательство. \\
 \raisebox{-1pt}[0pt][0pt]{\includegraphics[width=0.02\linewidth]{4.png}} Замечание (почему так важна формула обращения для характеристических функций?). \\
 \raisebox{-1pt}[0pt][0pt]{\includegraphics[width=0.02\linewidth]{3.png}} Следствие об устойчивости по суммированию (без док-ва). \\
 \raisebox{-1pt}[0pt][0pt]{\includegraphics[width=0.02\linewidth]{4.png}} Следствие об устойчивости по суммированию. Доказательство. \\
 \raisebox{-1pt}[0pt][0pt]{\includegraphics[width=0.02\linewidth]{3.png}} Теорема о непрерывном соответствие (без док-ва). \\
 \raisebox{-1pt}[0pt][0pt]{\includegraphics[width=0.02\linewidth]{3.png}} Теорема (закон больших чисел Хинчина) (без док-ва). \\
 \raisebox{-1pt}[0pt][0pt]{\includegraphics[width=0.02\linewidth]{4.png}}  Теорема (закон больших чисел Хинчина).  Доказательство. \\
 \raisebox{-1pt}[0pt][0pt]{\includegraphics[width=0.02\linewidth]{3.png}} Теорема (закон больших чисел Колмогорова) (без док-ва). \\
 \raisebox{-1pt}[0pt][0pt]{\includegraphics[width=0.02\linewidth]{5.png}} Теорема (закон больших чисел Колмогорова). Доказательство достаточности при 4-ом моменте. \\

 

\item[\S\, 4.4.] Центральная предельная теорема \\
 \raisebox{-1pt}[0pt][0pt]{\includegraphics[width=0.02\linewidth]{3.png}} Теорема (центральная предельная теорема) (без док-ва). \\
 \raisebox{-1pt}[0pt][0pt]{\includegraphics[width=0.02\linewidth]{4.png}} Теорема (центральная предельная теорема). Доказательство. \\        
 \raisebox{-1pt}[0pt][0pt]{\includegraphics[width=0.02\linewidth]{4.png}} Следствие (из ЦПТ). \\
 \raisebox{-1pt}[0pt][0pt]{\includegraphics[width=0.02\linewidth]{3.png}} Теорема (неравенство Берри-Эссеена) (без док-ва). \\
 \raisebox{-1pt}[0pt][0pt]{\includegraphics[width=0.02\linewidth]{4.png}} Замечание (о неулучшаемости неравенства Берри-Эссеена). \\
 \raisebox{-1pt}[0pt][0pt]{\includegraphics[width=0.02\linewidth]{5.png}} Теорема (оценка точности в теореме Пуассона). Доказательство. \\
    
    
\item[\S\, 3.9.] Условное математическое ожидание \\
 \raisebox{-1pt}[0pt][0pt]{\includegraphics[width=0.02\linewidth]{3.png}} Опр. (условного математического ожидания). \\
 \raisebox{-1pt}[0pt][0pt]{\includegraphics[width=0.02\linewidth]{3.png}} Теорема о существование УМО (без док-ва). \\
 \raisebox{-1pt}[0pt][0pt]{\includegraphics[width=0.02\linewidth]{4.png}} Свойства УМО с доказательством (УМО константы, УМО от измеримой с.в., монотонность,  линейность, неравенство треугольника,  аналог формулы полной вероятности). \\
 \raisebox{-1pt}[0pt][0pt]{\includegraphics[width=0.02\linewidth]{5.png}} Свойства УМО с доказательством (УМО по более бедной сигма алгебре, вынос измеримой с.в. ). \\
 \raisebox{-1pt}[0pt][0pt]{\includegraphics[width=0.02\linewidth]{5.png}} Теорема об ортогональной проекции. Доказательство. \\
 \raisebox{-1pt}[0pt][0pt]{\includegraphics[width=0.02\linewidth]{3.png}} Лемма (вычисление УМО для а.н.р.)  (без док-ва). \\
 \raisebox{-1pt}[0pt][0pt]{\includegraphics[width=0.02\linewidth]{4.png}} Лемма (вычисление УМО для а.н.р.). Доказательство. \\
 \raisebox{-1pt}[0pt][0pt]{\includegraphics[width=0.02\linewidth]{3.png}} Лемма (вычисление УМО для дискретных). (без док-ва). \\
 \raisebox{-1pt}[0pt][0pt]{\includegraphics[width=0.02\linewidth]{4.png}} Лемма (вычисление УМО для дискретных). Доказательство. \\
 \raisebox{-1pt}[0pt][0pt]{\includegraphics[width=0.02\linewidth]{5.png}} Теорема (УМО для гауссовских векторов). Доказательство. \\










    \end{enumerate}
    

\end{enumerate}

\end{document}